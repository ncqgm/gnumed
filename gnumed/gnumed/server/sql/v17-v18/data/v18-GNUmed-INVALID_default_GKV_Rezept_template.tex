% ----------------------------------------------------------------------
% Author: Christoph Becker, Karsten Hilbert
% License: GPL v2 or later
%
% **********************************************************************
% ************   WARNING   *********************************************
% **********************************************************************
%
% Note that the German legal situation DOES NOT ALLOW you to print
% prescriptions for "Kassenpatienten" with this GNUmed form template.
%
% **********************************************************************
% **********************************************************************
% **********************************************************************

\documentclass[10pt,german]{letter}

\usepackage[utf8x]{inputenc}
\usepackage[T1]{fontenc}
\usepackage{ucs}
\usepackage{babel}
\usepackage[abs]{overpic}
\usepackage[a4paper]{geometry}
\geometry{verbose}

% define page dimensions as needed
\pagestyle{empty}
\oddsidemargin 64mm		% (90mm vom linken Papierrand - 1 Inch LaTeX-Festrand links - 1mm Druckerrand)
\topmargin 27mm			% (52mm vom oberen Papierrand - 1 Inch LaTeX-Festrand oben)
\headheight 0mm
\headsep 0mm
\topskip 0mm
\footskip 0mm
%\textwidth146mm
%\textheight104mm
%\paperwidth148mm
%\paperheight105mm

% define where to put stuff
\newcount\ganzlinks
\newcount\stempeloffsetx
\newcount\stempeloffsety


\begin{document}
\setlength{\unitlength}{1mm}

% x-Koordinate tut nichts zur Sache (weil \oddsidemargin sie definiert)
% y-Koordinate schiebt d. 'Bild' nach unten
\begin{picture}(0,0)

	\ganzlinks=-80			% 80mm links der rechten unteren Ecke des Stammdatenkastens
	\stempeloffsetx=17
	\stempeloffsety=-80

	% KVK-Daten
	$<text_snippet::GKV_KVK_Daten-LaTeX_picture_environment//%s::9999>$

	% Medikamente
	\multiputlist(\ganzlinks,-10)(0,-5)[l]{
	**** FORMULAR DARF NICHT VERWENDET WERDEN ****,
	$<current_meds::%(substance)s {\footnotesize (%(brand)s)} %(preparation)s %(amount)s%(unit)s: %(schedule)s,//select::>$
	*************************************************,
	$<free_text::tex//Hinweis zum Rezept::9999>$
	}

	% Stempel
	$<text_snippet::LMcC-GKV_Stempel-LaTeX_picture_environment//%s::9999>$

\end{picture}


% Unterschrift
\begin{overpic}{$<data_snippet::1x1-transparent-PNG//%s//image/png//.png::250>$}
	\put(17,-70){$<data_snippet::autograph-LMcC//\includegraphics[width=30mm]{%s}//image/png//.png::250>$}
\end{overpic}

\end{document}

% ----------------------------------------------------------------------
% This file is *heavily* based on work by:
%
% Christoph Becker (Dauner Str. 21, 53539 Kelberg, Germany, cgbecker@gmx.de)
%
% contributed to GNUmed on 4. Febr. 2005 
%
% Prinzip:
%
% Das gesamte Dokument besteht nur aus EINEM Bild (d.h. einer
% 'picture'-Umgebung).
%
% Alle Felder des Formulars werden als Bildelemente vom Typ Text
% platziert (Siehe Helmut Kopka, Latex - Eine Einfuehrung, Bd 1,
% Kap. 6, Bilder) und zwar relativ, bezogen auf die Ursprungs-
% koordinate des Bildes. Als Ursprungskoordinate wird die rechte
% untere Ecke des Rahmens mit den Versichertenkartendaten
% (Formularkopf) verwendet.
%
% Diese Ursprungskoordinate kann nicht an einem Papierrand liegen, weil
% LaTeX einen Seitenrand von 0 ignoriert. Daher ist es besser, bei
% Formularen einen leicht erkennbaren Ursprungspunkt im Formular zu
% benutzen. Die Bildgroesse kann (0,0) sein.
%
% Der KZV- bzw. KV-Stempel wird mit einem Offset eingestellt, hier:
%
% \newcount\stempeloffsetx \stempeloffsetx=20
% \newcount\stempeloffsety \stempeloffsety=-10
%
% Dadurch kann der 'Stempel' durch einfaches Kopieren auch in andere
% Kassenformulare uebernommen werden (AU, HKP usw.). Es muss dann
% lediglich der Offset geaendert werden.
%
% Das Formular wurde ursprünglich auf einem EPSON 630 LQ-Nadeldrucker
% unter Windows XP mit TeXnicCenter entwickelt. Wenn sich auf einem
% anderen Drucker oder unter Linux zeigt, dass die 
% Ursprungskoordinate nicht stimmt, so kann diese ueber
%
%	\oddsidemargin
%	\topmargin
%
% angepasst werden.
% ----------------------------------------------------------------------
